\chapter{TuMag's design and calibration.}

In this chapter we take the first steps of the journey of developing an instrument to observe the Sun. We will define... 

Fabry-Pérot interferometers (FPIs) are widely employed in the field of solar physics. Their spectroscopic and tunability properties make them especially suitable for selecting a narrow spectral band of incoming light. They also offer a two-dimensional view of the solar scene, hence allowing for the implementation of powerful and widespread image post-processing reconstruction techniques, such as phase diversity \citep{PD_etalon} and multi-object multi-frame blind deconvolution (MOMFBD; \citealt{mombfd}), which are difficult to implement in slit-based spectrographs (\citealt{image_spectro}, \citealt{image_spectro_2}). Many state-of-the-art instruments use FPIs as narrowband tunable filters. Among others, these instruments include the spaceborne Polarimetric and Helioseismic Imager \citep[][]{PHI} aboard the Solar Orbiter mission \citep[][]{SO} (SO/PHI); the Imaging Magnetograph Experiment (IMaX) instrument \citep[][]{IMaX}, which flew on the first two flights of the balloon-born SUNRISE observatory (\citealt{SunriseI}, \citealt{SunriseII}); and the Tunable Magnetograph (TuMag) instrument for its third edition. These instruments are based on solid LiNbO$_3$ etalons. Regarding ground-based instruments, some examples include the Crisp Imaging Spectro-Polarimeter (CRISP) at the Swedish 1-m Solar telescope \citep[][]{crisp} at the Observatorio del Roque de los Muchachos in La Palma, Canary Islands; the GREGOR Fabry-Perot Interferometer (GFPI; \citealt{GFPI}, \citealt{GREGOR}) at the Observatorio del Teide in Tenerife, Canary Islands; the Visible Tunable Filter \citep[VTF;][]{VTF} developed for the \textit{Daniel K. Inouye} Solar Telescope \citep[DKIST;][]{DKIST} of the Haleakal\=a Observatory in Hawaii; and the future Tunable Imaging Spectropolarimeter (TIS) of the European Solar Telescope \citep{EST}, all of which are based on air-gapped etalons. 

The SUNRISE III mission aims to study and stablish the relations and couplings between the phenomena ocurring at different layers of the Sun's surface. With this purpose in mind, three different post-focal instruments were included in the design, each of them responsible of observing at different regions of the spectrum. The SUNRISE UV Spectropolarimeter and Imager (SUSI, \textbf{REFERENCIA}), which will observe the spectra between 309 nm and 417 nm; The Sunrise Chromospheric Infrared spectroPolarimeter (SCIP, \textbf{REFERENCIA}), which will observe the near-infrared; and lastly, the Tunable Magnetograph (TuMag), which will observe three spectral lines in the visible, at 525.02 nm, 525.06 nm and 517 nm. 

The design from scratch of an instrument such as this is very complex. There are many things that have to be meticulously designed and tested which span many fields of expertise, like optics, electronics, software, hardware, or thermal design. To avoid undue extension of this thesis, we will focus on the aspects of the design directly related to the optical properties, that is, regarding the spectral, imaging and polarimetric capabilities of the instrument. 

\section{The Tunable magnetograph: TuMag}

The Tunable Magnetograph (TuMag) is a tunable imaging spectropolarimeter specifically designed to deliver high spatial resolution images across multiple spectral lines in four distinct polarization states. Consequently, TuMag is capable of measuring the four Stokes parameters, thus enabling the inference of the three components of the magnetic field and the LOS velocities at each observed spectral line. Moreover, this data must be acquired following a series of strict requiremets regarding optical quality, polarimetric efficiencies, required SNR, spectral performance and time limitations. A summary of these requirements is provided in Table \ref{table: Tumags requirements}. 

\begin{table}
    \centering
   \begin{tabular}{cc}
    \hline
    \hline
    Requirements & Value \\
    \hline
    Field of view & 63'' x 63'' \\
    RMS wavefront error & $W \approx \lambda / 14$\\
    Spatial sampling & $3 \times 3 $ pixels \\
    Plate scale & 0.0378'' / pixel \\
    Polarimetric efficiencies & $\epsilon _ {1, 2, 3} \lessapprox \frac{1}{\sqrt{3}}$\\
    SNR ratio & $\left(\frac{S}{N}\right) _ 0 \gtrapprox 1700$ \\
    Spectral resolution & $< 9$ pm\\  
    Spectral lines & Fe I 5250.2 \r{A}, Fe I 5250.6 \r{A} and Mg I $b_2$ 5172.7 \r{A}. \\
    Time for a two-line observation & $< 90$ s\\
    \hline
    \hline
    \end{tabular}
    \caption{Tumag scientific requirements.}
    \label{table: Tumags requirements}
\end{table}

\subsection{Imaging performance.}

TuMag measures the photons through two custom-made cameras \citep{tumag-cams} based on the GPIXEL back-illuminated GSENSE400BSI $2k \times 2k$ pixel detector. The reason for the existance of two cameras instead of one is the dual-beam  

