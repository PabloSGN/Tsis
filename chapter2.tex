\chapter{TuMag's design and calibration.}

In this chapter we take the first steps of the journey of developing an instrument to observe the Sun. We will define... 

Fabry-Pérot interferometers (FPIs) are widely employed in the field of solar physics. Their spectroscopic and tunability properties make them especially suitable for selecting a narrow spectral band of incoming light. They also offer a two-dimensional view of the solar scene, hence allowing for the implementation of powerful and widespread image post-processing reconstruction techniques, such as phase diversity \citep{PD_etalon} and multi-object multi-frame blind deconvolution (MOMFBD; \citealt{mombfd}), which are difficult to implement in slit-based spectrographs (\citealt{image_spectro}, \citealt{image_spectro_2}). Many state-of-the-art instruments use FPIs as narrowband tunable filters. Among others, these instruments include the spaceborne Polarimetric and Helioseismic Imager \citep[][]{PHI} aboard the Solar Orbiter mission \citep[][]{SO} (SO/PHI); the Imaging Magnetograph Experiment (IMaX) instrument \citep[][]{IMaX}, which flew on the first two flights of the balloon-born SUNRISE observatory (\citealt{SunriseI}, \citealt{SunriseII}); and the Tunable Magnetograph (TuMag) instrument for its third edition. These instruments are based on solid LiNbO$_3$ etalons. Regarding ground-based instruments, some examples include the Crisp Imaging Spectro-Polarimeter (CRISP) at the Swedish 1-m Solar telescope \citep[][]{crisp} at the Observatorio del Roque de los Muchachos in La Palma, Canary Islands; the GREGOR Fabry-Perot Interferometer (GFPI; \citealt{GFPI}, \citealt{GREGOR}) at the Observatorio del Teide in Tenerife, Canary Islands; the Visible Tunable Filter \citep[VTF;][]{VTF} developed for the \textit{Daniel K. Inouye} Solar Telescope \citep[DKIST;][]{DKIST} of the Haleakal\=a Observatory in Hawaii; and the future Tunable Imaging Spectropolarimeter (TIS) of the European Solar Telescope \citep{EST}, all of which are based on air-gapped etalons. 

The SUNRISE III mission aims to study and stablish the relations and couplings between the phenomena ocurring at different layers of the Sun's surface. With this purpose in mind, three different post-focal instruments were included in the design, each of them responsible of observing at different regions of the spectrum. The SUNRISE UV Spectropolarimeter and Imager (SUSI, \textbf{REFERENCIA}), which will observe the spectra between 309 nm and 417 nm; The Sunrise Chromospheric Infrared spectroPolarimeter (SCIP, \textbf{REFERENCIA}), which will observe the near-infrared; and lastly, the Tunable Magnetograph (TuMag), which will observe three spectral lines in the visible, at 525.02 nm, 525.06 nm and 517 nm. 

The design from scratch of an instrument such as this is very complex. There are many things that have to be meticulously designed and tested which span many fields of expertise, like optics, electronics, software, hardware, or thermal design. To avoid undue extension of this thesis, we will focus on the aspects of the design directly related to the optical properties, that is, regarding the spectral, imaging and polarimetric capabilities of the instrument. 

\section{The Tunable magnetograph: TuMag}

The Tunable Magnetograph (TuMag) is a tunable imaging spectropolarimeter designed to deliver high spatial resolution images across multiple spectral lines in four distinct polarization states. Consequently, TuMag is capable of measuring the four Stokes parameters, thus enabling the inference of the three components of the magnetic field and the LOS velocities at all the selected spectral lines. Moreover, this data must be acquired following a series of strict requiremets regarding optical quality, polarimetric efficiencies, required SNR, spectral performance and time limitations. A summary of these requirements is provided in Table \ref{table: Tumags requirements}. 

\begin{table}
    \centering
   \begin{tabular}{cc}
    \hline
    \hline
    Requirements & Value \\
    \hline
    Field of view & $63''$ x $63''$ \\
    RMS wavefront error & $W \sim \lambda / 14$\\
    Spatial sampling & $3 \times 3 $ pixels \\
    Plate scale & $0.0378''$ / pixel \\
    Polarimetric efficiencies & $\epsilon _ {1, 2, 3} \lessapprox \frac{1}{\sqrt{3}}$\\
    SNR ratio & $\left(\frac{S}{N}\right) _ 0 \gtrapprox 1700$ \\
    Spectral resolution & $< 9$ pm\\  
    Spectral lines & Fe I 5250.2 \r{A}, Fe I 5250.6 \r{A} and Mg I $b_2$ 5172.7 \r{A}. \\
    Time for a two-line observation & $< 90$ s\\
    \hline
    \hline
    \end{tabular}
    \caption{Tumag scientific requirements.}
    \label{table: Tumags requirements}
\end{table}

\subsection{TuMag's design.}

Light is delivered to TuMag by the ISLiD system and subsequently re-imaged onto two cameras where the images are recorded. Before reaching the cameras, the light passes through all the different subsystems of the optical unit. The first components encountered by the light are a blocking prefilter and a the filter wheels. The blocking prefilter, with a wide bandpass centered at 520 nm, is employed to eliminate unwanted spectral ranges. The filter wheels  are comprised by a double-disk system \citep{filter-wheels} that houses the prefilters for selecting specific spectral lines and a series of calibration modules. Specifically, three prefilters are mounted on the second disk of the filter wheel, corresponding to the spectral lines Fe I 5250.2 \r{A}, Fe I 5250.6 \r{A}, and Mg I $b_2$ 5172.7 \r{A}. A detailed overview of the spectral properties of these prefilters will be provided in Section \textcolor{red}{XX}. Additionally, the filter wheel includes a PD plate, which is used to introduce a known defocus into the final image to facilitate image reconstruction techniques, along with a linear polarizer, a plate of micropolarizers, and a pinhole set.

After passing through the filter wheels, the light is directed into the Polarization Modulation Package (PMP), a subsystem derived from the SO/PHI instrument (\citealt{pmp1}, \citealt{PHI}). The PMP's primary function is to modulate the light to produce the different polarization states required to deduce the Stokes components. This is achieved using two liquid crystal variable retarders (LCVRs), which are oriented with their fast axes at 45$^\circ$ relative to each other. These LCVRs induce a retardance on the transmitted light that varies with the voltage applied across the crystals. The system can operate in two distinct modulation schemes: a vector modulation scheme, which generates four independent linear combinations of equally-weighted Stokes components across consecutive observations, allowing for the retrieval of the full Stokes vector after demodulation; and a longitudinal modulation scheme, which generates only two modulations, providing information solely on the intensity and circular polarization.

Following modulation, the light is directed into a $LiNbO_3$ Fabry-Pérot etalon, that likewise IMaX, is in a collimated setup and with a double pass configuration \citep{etalon-doublepass}. In this setup, after traversing the etalon once, the light is redirected by a pair of mirrors to pass through the etalon a second time. This double-pass configuration significantly enhances spectral resolution by narrowing the transmission profile. The $LiNbO_3$ etalon achieves wavelength tuning by varying the refractive index of the cavity through the application of high voltages (ranging from $-4000$ V to $4000$ V) to the mirrors. Compared to air-gapped etalons, $LiNbO_3$ etalons offer the advantage of having no moving parts, which is particularly beneficial for spaceborne or balloon-borne instruments. However, this advantage comes with the need for precautions to prevent discharges caused by air ionization.

The last element light encounters before reaching the cameras is a polarizing beam-splitter. In this point the light beam is divided into its two orthogonal components of linear polarization, and each of these beams is directed towrds a different camera.    




\subsection{Imaging performance.}

TuMag captures photons using two custom-made cameras \citep{tumag-cams} equipped with GPIXEL back-illuminated GSENSE400BSI detectors, each featuring a $2k \times 2k$ pixel array. The use of two cameras, rather than a single one, is required by the dual-beam polarimetric scheme, in which each camera records \textcolor{red}{two opposed orthogonal states of polarization}. These cameras provide a broad FoV of $63'' \times 63''$, sufficient to encompass an entire medium-sized active region, with a plate scale of $0.0378''$/pixel.

In order to fulfill the requirement of the wavefront error of $W \sim \lambda / 14$, the instrument must have means to correct for the additional aberrations introduced by the telescope, the image stabilization and light distribution (ISLiD) system and uncorrected jittering. For this purpose, TuMag is equipped with PD plates that allow to aply image reconstruction  techniques to the final images. 



